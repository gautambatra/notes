\documentclass[11pt,twoside,a4paper,titlepage]{article}
\usepackage{xcolor}
\usepackage{soul}
\usepackage[T1]{fontenc}

\begin{document}
\title{SystemVerilog Notes}
\author{Gautam Batra}
\date{\today}
\maketitle


\section{About SystemVerilog/Features}
\begin{itemize}
    \item Loosely typed
    \item Variable types:
        \begin{itemize}
            \item Register
            \item Integer
            \item Time 
            \item Real
        \end{itemize}
    \item Net type determines how it's resolved if it has multiple drivers
    \item Several transistor and gate level primitives
    \item Construct available for user defined primitives
    \item Programmable Logic Array (PLA)  modelling mechanism
    \item Propagation delay for nets and primitives
    \item Drive strength for continuous assignments and primitives
    \item Path delays, pulse filtering and timing checks for modules
    \item Comprehensive, i.e., no user extensions
    \item Description of HW as set of modules
    \item Can instantiate one module in another (creates a copy)
    \item 2 ways of specifying module:
        \begin{enumerate}
            \item Behavioral representation: program-like (starting point)
            \item Structural representation: internal logic structure
        \end{enumerate}
    \item Modules interconnected by nets (like wires)
    \item What we can do with SystemVerilog code
        \begin{itemize}
            \item Simulate system and verify the operation
                \begin{itemize}
                    \item Use \emph{test benches} to provide input and verify output
                \end{itemize}
            \item Use synthesis tool to map to HW
                \begin{itemize}
                    \item Converts it into netlist of low-level primitives
                    \item HW could be ASIC or FPGA
                    \item Don’t need testbenches, can directly test using real signals
                    \item ASIC: high performance, high packing density, used in large numbers, expensivePGA: fast turnaround time, flexible, performance trade-off
                \end{itemize}
        \end{itemize}
    \item \textbf{initial} : block executed only once
    \item \textbf{\$monitor} : like printer (prints whenever any of the variables in brackets change)
    \item \%b : bit
    \item \textbf{\$dumpfile}: dump testbench changes to a vcd (value change dump) file
    \item \textbf{\$dumpvars (0, testbenchModule)}: All variables to be dumped.
    \item Test Bench
        \begin{itemize}
            \item \hl{the variables in the initial block that appear on LHS (ie, whose values we are changing) need to be declared as reg}
            \item \hl{Other variables to be declared as wire}
        \end{itemize}
    \item Icarus verilog
        \begin{itemize}
            \item Iverilog -o <output\_file> source.v testbench.v \textbar p <output\_file>  //runs the simulation and prints results
        \end{itemize}
\end{itemize}
\section{signals}
\begin{itemize}
    \item SystemVerilog supports 4 value levels
        \begin{enumerate}
            \item 0
            \item 1
            \item x : unknown (all registers initialized to x)
            \item z : high impedance (all unconnected nets)
        \end{enumerate}
    \item Signal strengths (in descending order):\\
        \begin{center}
            \begin{tabular}[c]{|l|l|} \hline
                \textbf{Strength}  & \textbf{Type} \\
                \hline
                supply    &  Driving \\
                \hline
                strong    &  Driving \\
                \hline
                pull      &  Driving \\
                \hline
                large     &  Storage \\
                \hline
                weak      &  Driving \\
                \hline
                medium    &  Storage \\
                \hline
                small     &  Storage \\
                \hline
                highz     &  High Impedance \\
                \hline
            \end{tabular}
        \end{center}
        \begin{itemize}
            \item If 2 signals get driven on a wire, stronger signal prevails
            \item useful for MOS level circuits eg: dynamic MOS
        \end{itemize}
\end{itemize}
\section{Data Types}
\begin{enumerate}
    \item Net
        \begin{itemize}
            \item \hl{Must be continuously driven}
            \item \hl{Can't store a value}
            \item Represents connection b/w HW elements
            \item During synthesis always maps to a wire
            \item 1 bit value by default
            \item Default value is 'z' (high impedance state)
                \begin{itemize}
                    \item Not driven, neither 1 nor 0
                \end{itemize}
            \item Typically used to model connections b/w continuous assignments and instantiations 
            \item Different types:
                \begin{itemize}
                    \item wire, wor, wand, tri, supply0, supply1, etc.
                    \item wire and tri are equivalent
                        \begin{itemize}
                            \item Wire can only be driven using assign statements
                        \end{itemize}
                    \item wor and wand insert OR and AND respectively at the connection
                    \item supply0 and supply1 model power supply connections
                \end{itemize}
        \end{itemize}
    \item Register
        \begin{itemize}
            \item \hl{Retains/holds last value assigned}
            \item \hl{Typically used to represent storage elements, sometimes can translate to combinational circuits also}
            \item X = A + B // \hl{no 'assign'}
            \item During synthesis, maps either to 'wire' or a 'storage cell' depending on the context of the assignment
            \item Not to be confused with a hardware register, ie flip-flop
            \item Register types in SystemVerilog \\
                \begin{center}
                    \begin{tabular}[c]{|l|l|} \hline
                        reg    &  most common \\
                        \hline
                        integer  &  used for counting \\
                        \hline
                        real     &  floating point numbers \\
                        \hline
                        time     &  simulation time (not used in synthesis) \\
                        \hline
                    \end{tabular}
                \end{center}
        \end{itemize}
        \begin{enumerate}
            \item reg
                \begin{itemize}
                    \item default value = x
                    \item can be assigned in synchronism with a clock or otherwise
                    \item Treated as unsigned number
                    \item \hl{Used to model actual sequential HW components like counters, shift registers, etc.}
                    \item \hl{Can only be driven in procedural blocks like always and initial}
                \end{itemize}
            \item integer
                \begin{itemize}
                    \item signed
                    \item default size is 32 bits (can't specify size)
                    \item Synthesis tool determines the size using data flow analysis
                \end{itemize}
            \item real
                \begin{itemize}
                    \item floating point numbers
                    \item if assigned to an integer, rounded off
                \end{itemize}
            \item time
                \begin{itemize}
                    \item store simulation time
                    \item \emph{\$time} gives current simulation time 
                \end{itemize}
        \end{enumerate}
\end{enumerate}
\section{Module}
\begin{itemize}
    \item $\sim$ Function
    \item Basic unit of HW
    \item Can't contain definition of other modules
        \begin{itemize}
            \item Other modules can be instantiated inside a module
            \item A copy of the instantiated module is embedded in the main (instantiating) module for each instantiation
        \end{itemize}
    \item Allows creation of hierarchy
\end{itemize}

\end{document}
